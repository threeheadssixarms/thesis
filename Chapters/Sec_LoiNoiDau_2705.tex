Tính toán tiến hóa là một lĩnh vực quan trọng trong khoa học máy tính. Các giải thuật tiến hóa có thể tìm được lời giải xấp xỉ cho nhiều bài toán không thể được giải bằng những cách thông thường. Đôi khi, việc tìm ra một lời giải chính xác đòi hỏi thời gian và công sức quá lớn khiến cho việc giải gần như là bất khả thi, trong khi một lời giải gần đúng cũng có thể đáp ứng được các yêu cầu trong thực tế. Khi đó, các giải thuật tiến hóa có thể trở nên rất hiệu quả vì chúng tìm được lời giải “tốt nhất” trong các lời giải có thể tìm được, thay vì cố gắng vô ích trong việc tìm kiếm một lời giải hoàn hảo.
Tuy nhiên, phần lớn các giải thuật tiến hóa được xây dựng để giải một bài toán đơn lẻ. Sự xuất hiện của giải thuật tiến hóa đa nhân tố (MFEA) đã mở ra những tiềm năng mới cho tính toán tiến hóa để nghiên cứu và khai thác. Giải thuật MFEA cơ bản xuất hiện trong những năm gần đây cho phép đồng thời giải quyết hai hay nhiều bài toán tối ưu. Các ứng dụng của giải thuật chưa được nghiên cứu sâu rộng nhưng những kết quả ban đầu đã hứa hẹn cho sự cải thiện về chất lượng lời giải so với các giải thuật đơn nhiệm hoặc khả năng rút ngắn thời gian chạy so với tổng thời gian giải lần lượt từng bài toán. 
Đồ án sẽ tìm hiểu về giải thuật MFEA và áp dụng giải thuật cho hai bài toán tối ưu trên đồ thị. Sau khi trình bày cơ sở lý thuyết về giải thuật, đồ án đề xuất phương pháp cài đặt giải thuật và trình bày các kết quả thu được sau khi tổng hợp. Đồ án hy vọng sẽ đem lại những kết quả chính xác và rút ra được những kết luận có ích cho việc phát triển thuật toán MFEA cho các bài toán đồ thị sau này.
Bài toán Cây khung chi phí định tuyến trên đồ thị phân cụm (Minimum Routing Cost Clustered Spanning Tree problem) là một biến thể của bài toán nổi tiếng Cây khung chi phí định tuyến (Minimum Routing Cost Tree). Bài toán có nhiều ứng dụng thực tế trong phân loại trong sinh vật học, phân phối hàng hóa trong các mạng lưới phân phối, truyền tải trong mạng viễn thông, v.v.
Bài toán Cây phân cụm đường đi ngắn nhất (Clustered Shortest Path Tree problem) có mục tiêu là tìm ra một cây khung phân cụm của đồ thị đầy đủ đầu vào sao cho tổng khoảng cách giữa các đỉnh tới đỉnh nguồn là nhỏ nhất. Bài toán có thể áp dụng cho tối ưu các mạng có một đỉnh đóng vai trò quan trọng nhất trong mạng, ví dụ như là kho hàng trong một hệ thống phân phối hàng hóa hay nguồn nước trong một hệ thống tưới tiêu.
