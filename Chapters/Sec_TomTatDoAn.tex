\fontsize{12}{16}\fontfamily{cmr}\selectfont
\glsresetall

Giải thuật tiến hóa đa nhân tố (Multifactorial Evolutionary Algorithm - MFEA) là một giải thuật mới được biết đến trong những năm gần đây, hứa hẹn nhiều hướng đi và đóng góp mới đối với lĩnh vực tính toán tiến hóa. Mặc dù mang rất nhiều tiềm năng như vậy nhưng giải thuật mới chỉ được nghiên cứu để áp dụng cho rất ít bài toán về đồ thị. Do đó, đồ án này nghiên cứu về giải thuật MFEA cơ bản và đề xuất một giải thuật tiến hóa đa nhân tố CluMFEA đa nhiệm để giải hai bài toán về đồ thị rất nổi tiếng: bài toán Cây khung phân cụm với chi phí định tuyến nhỏ nhất và bài toán Cây phân cụm đường đi ngắn nhất. Ngoài ra, đồ án cũng đề xuất một giải thuật di truyền CluGA đơn nhiệm để giải từng bài toán trên. Đồ án cài đặt chương trình thử nghiệm cho hai giải thuật đề xuất và trình bày những kết quả so sánh, đánh giá, kết luận về hai thuật toán. Đặc biệt, đồ án sẽ chỉ ra một yếu tố quan trọng quyết định CluMFEA hay CluGA đạt được kết quả tốt khi giải những bài toán này.

Đồ án được tổ chức như sau:
\begin{itemize}
	\item \textbf{Chương 1} trình bày về các nội dung lý thuyết làm cơ sở cho đồ án như các giải thuật di truyền, giải thuật \gls{mfea} cơ bản và một số định nghĩa quan trọng của lý thuyết đồ thị.
	\item \textbf{Chương 2} giới thiệu bài toán  \gls{clumrct}, \gls{cstp}. Phần này bao gồm  những ứng dụng và các nghiên cứu liên quan của hai bài toán.
	\item \textbf{Chương 3} đề xuất một giải thuật tiến hóa đa nhân tố CluMFEA và một giải thuật di truyền CluGA cho hai bài toán được trình bày ở Chương 2.
	\item \textbf{Chương 4} trình bày kết quả thử nghiệm và cung cấp phân tích hiệu suất của các giải thuật đã đề xuất.
\end{itemize}
