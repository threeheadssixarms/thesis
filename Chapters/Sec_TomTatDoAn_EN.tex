\fontsize{12}{16}\fontfamily{cmr}\selectfont

The Multifactorial Evolutionary Algorithm (MFEA) is a recently proposed algorithm that points out a new direction in the evolutionary computing field. Much as the algorithm seems promising, only a few of its possible applications in the solving of graph problems have been examined. Hence, in this thesis, I would like to devise an MFEA which is novel for its representation of chromosomes and its genetic operators, for two well-known graph problems, the Minimum Routing Cost Clustered Spanning Tree problem and the Clustered Shortest Path Tree problem. A genetic algorithm utilizing the same representation and genetic operators is also proposed in this thesis. The two proposed algorithms are implemented and compared. Most importantly, the thesis points out an important influencing factor that decides which algorithm performs better on each input instance.

The thesis is organized as follows:

\begin{itemize}
	\item \textbf{Chapter 1}  provides a brief review of genetic algorithms, the basic MFEA, and some important definitions related to graph theory.
	\item \textbf{Chapter 2} introduces the Minimum Routing Cost Clustered Spanning Tree problem, the Clustered Shortest Path Tree problem, their applications, and related work.
	\item \textbf{Chapter 3} proposes an MFEA, namely CluMFEA, and a genetic algorithm, namely CluGA to solve the two problems explained in Chapter 2, and presents other methods used in the execution of the algorithms.
	\item \textbf{Chapter 4} presents experimental results and provides an analysis of the proposed algorithms' performance.
\end{itemize}