\fontsize{12}{16}\fontfamily{cmr}\selectfont

Trong những năm gần đây, lớp các bài toán về cây khung phân cụm được áp dụng rộng rãi trong rất nhiều lĩnh vực như phân phối hàng hóa, truyền tải viễn thông, phân loại trong sinh vật học,… Bài toán Cây khung phân cụm với chi phí định tuyến nhỏ nhất (Minimum Routing Cost Clustered Spanning Tree - CluMRCT) và bài toán Cây phân cụm đường đi ngắn nhất (Clustered Shortest Path Tree – CluSPT) là hai trong số những bài toán như vậy. Là một biến thể của bài toán Cây khung với chi phí định tuyến nhỏ nhất (Minimum Routing Cost Tree), bài toán CluMRCT giúp tối ưu thiết kế mạng của những hệ thống có các điểm đầu cuối được nhóm thành các cụm và chú trọng việc kết nối giữa các điểm trong cùng một cụm. Trong khi đó, bài toán CluSPT giúp tối ưu các mạng với một đỉnh có vai trò quan trọng hơn tất cả các đỉnh khác, như kho hàng trong hệ thống phân phối hàng hóa hay nguồn nước trong một hệ thống tưới tiêu.

Nhiều hướng tiếp cận để giải hai bài toán trên đã được thử nghiệm. Tuy nhiên, những nghiên cứu áp dụng tính toán tiến hóa – một lĩnh vực quan trọng trong khoa học máy tính - cho hai bài toán còn rất hiếm. Các giải thuật tiến hóa đem lại rất nhiều lợi ích cho các bài toán tối ưu đòi hỏi nhiều thời gian để giải chính xác, vì chúng có thể tìm được lời giải gần đúng trong một khoảng thời gian ngắn hơn. 

Tuy nhiên, phần lớn các giải thuật tiến hóa được xây dựng để giải một bài toán đơn lẻ. Sự xuất hiện của giải thuật tiến hóa đa nhân tố (MFEA) đã mở ra những tiềm năng mới để nghiên cứu và khai thác cho tính toán tiến hóa. Giải thuật MFEA cơ bản xuất hiện trong những năm gần đây cho phép đồng thời giải quyết hai hay nhiều bài toán tối ưu. Các ứng dụng của giải thuật MFEA chưa được nghiên cứu sâu rộng, nhưng những kết quả ban đầu rất hứa hẹn về chất lượng lời giải so với các giải thuật đơn nhiệm hoặc về thời gian giải so với tổng thời gian các giải thuật đơn nhiệm giải lần lượt từng bài toán. 

Đồ án sẽ tìm hiểu về giải thuật MFEA cơ bản và đề xuất một giải thuật MFEA để giải đồng thời bài toán CluSPT và CluMRCT, được gọi là giải thuật CluMFEA. Ngoài ra, đồ án đề xuất một giải thuật di truyền CluGA cho mỗi bài toán. Sau đó, đồ án sẽ cài đặt để so sánh cho hai giải thuật và trình bày các kết quả thu được sau khi tổng hợp, phân tích, đánh giá. Đồ án hy vọng sẽ đem lại những kết quả tích cực và rút ra được những kết luận có ích cho việc phát triển thuật toán MFEA cho các bài toán đồ thị sau này.
